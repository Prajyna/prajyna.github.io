% Options for packages loaded elsewhere
% Options for packages loaded elsewhere
\PassOptionsToPackage{unicode}{hyperref}
\PassOptionsToPackage{hyphens}{url}
\PassOptionsToPackage{dvipsnames,svgnames,x11names}{xcolor}
%
\documentclass[
  10pt,
]{article}
\usepackage{xcolor}
\usepackage[top=15mm,left=25mm,right=25mm,bottom=25mm,heightrounded]{geometry}
\usepackage{amsmath,amssymb}
\setcounter{secnumdepth}{-\maxdimen} % remove section numbering
\usepackage{iftex}
\ifPDFTeX
  \usepackage[T1]{fontenc}
  \usepackage[utf8]{inputenc}
  \usepackage{textcomp} % provide euro and other symbols
\else % if luatex or xetex
  \usepackage{unicode-math} % this also loads fontspec
  \defaultfontfeatures{Scale=MatchLowercase}
  \defaultfontfeatures[\rmfamily]{Ligatures=TeX,Scale=1}
\fi
\usepackage[]{libertinus}
\ifPDFTeX\else
  % xetex/luatex font selection
\fi
% Use upquote if available, for straight quotes in verbatim environments
\IfFileExists{upquote.sty}{\usepackage{upquote}}{}
\IfFileExists{microtype.sty}{% use microtype if available
  \usepackage[]{microtype}
  \UseMicrotypeSet[protrusion]{basicmath} % disable protrusion for tt fonts
}{}
\makeatletter
\@ifundefined{KOMAClassName}{% if non-KOMA class
  \IfFileExists{parskip.sty}{%
    \usepackage{parskip}
  }{% else
    \setlength{\parindent}{0pt}
    \setlength{\parskip}{6pt plus 2pt minus 1pt}}
}{% if KOMA class
  \KOMAoptions{parskip=half}}
\makeatother
% Make \paragraph and \subparagraph free-standing
\makeatletter
\ifx\paragraph\undefined\else
  \let\oldparagraph\paragraph
  \renewcommand{\paragraph}{
    \@ifstar
      \xxxParagraphStar
      \xxxParagraphNoStar
  }
  \newcommand{\xxxParagraphStar}[1]{\oldparagraph*{#1}\mbox{}}
  \newcommand{\xxxParagraphNoStar}[1]{\oldparagraph{#1}\mbox{}}
\fi
\ifx\subparagraph\undefined\else
  \let\oldsubparagraph\subparagraph
  \renewcommand{\subparagraph}{
    \@ifstar
      \xxxSubParagraphStar
      \xxxSubParagraphNoStar
  }
  \newcommand{\xxxSubParagraphStar}[1]{\oldsubparagraph*{#1}\mbox{}}
  \newcommand{\xxxSubParagraphNoStar}[1]{\oldsubparagraph{#1}\mbox{}}
\fi
\makeatother


\usepackage{longtable,booktabs,array}
\usepackage{calc} % for calculating minipage widths
% Correct order of tables after \paragraph or \subparagraph
\usepackage{etoolbox}
\makeatletter
\patchcmd\longtable{\par}{\if@noskipsec\mbox{}\fi\par}{}{}
\makeatother
% Allow footnotes in longtable head/foot
\IfFileExists{footnotehyper.sty}{\usepackage{footnotehyper}}{\usepackage{footnote}}
\makesavenoteenv{longtable}
\usepackage{graphicx}
\makeatletter
\newsavebox\pandoc@box
\newcommand*\pandocbounded[1]{% scales image to fit in text height/width
  \sbox\pandoc@box{#1}%
  \Gscale@div\@tempa{\textheight}{\dimexpr\ht\pandoc@box+\dp\pandoc@box\relax}%
  \Gscale@div\@tempb{\linewidth}{\wd\pandoc@box}%
  \ifdim\@tempb\p@<\@tempa\p@\let\@tempa\@tempb\fi% select the smaller of both
  \ifdim\@tempa\p@<\p@\scalebox{\@tempa}{\usebox\pandoc@box}%
  \else\usebox{\pandoc@box}%
  \fi%
}
% Set default figure placement to htbp
\def\fps@figure{htbp}
\makeatother

\ifLuaTeX
  \usepackage{luacolor}
  \usepackage[soul]{lua-ul}
\else
  \usepackage{soul}
\fi




\setlength{\emergencystretch}{3em} % prevent overfull lines

\providecommand{\tightlist}{%
  \setlength{\itemsep}{0pt}\setlength{\parskip}{0pt}}



 


\usepackage{setspace}
\setstretch{1}
\usepackage{fancyhdr}
\usepackage{multicol}
\usepackage{lipsum}
\usepackage{hyperref}
\pagestyle{fancy}
\fancyhf{} % clear all header/footer fields
\rfoot{\thepage} % right footer with page number
\setlength{\multicolsep}{2pt plus 1.0pt minus 0.75pt}
\setlength{\columnsep}{3cm}
\usepackage{titlesec}
\titlespacing*{\section}{0pt}{0.5em}{0.3em}
\titlespacing*{\subsection}{0pt}{0.3em}{0.2em}
\makeatletter
\@ifpackageloaded{caption}{}{\usepackage{caption}}
\AtBeginDocument{%
\ifdefined\contentsname
  \renewcommand*\contentsname{Table of contents}
\else
  \newcommand\contentsname{Table of contents}
\fi
\ifdefined\listfigurename
  \renewcommand*\listfigurename{List of Figures}
\else
  \newcommand\listfigurename{List of Figures}
\fi
\ifdefined\listtablename
  \renewcommand*\listtablename{List of Tables}
\else
  \newcommand\listtablename{List of Tables}
\fi
\ifdefined\figurename
  \renewcommand*\figurename{Figure}
\else
  \newcommand\figurename{Figure}
\fi
\ifdefined\tablename
  \renewcommand*\tablename{Table}
\else
  \newcommand\tablename{Table}
\fi
}
\@ifpackageloaded{float}{}{\usepackage{float}}
\floatstyle{ruled}
\@ifundefined{c@chapter}{\newfloat{codelisting}{h}{lop}}{\newfloat{codelisting}{h}{lop}[chapter]}
\floatname{codelisting}{Listing}
\newcommand*\listoflistings{\listof{codelisting}{List of Listings}}
\makeatother
\makeatletter
\makeatother
\makeatletter
\@ifpackageloaded{caption}{}{\usepackage{caption}}
\@ifpackageloaded{subcaption}{}{\usepackage{subcaption}}
\makeatother
\usepackage{bookmark}
\IfFileExists{xurl.sty}{\usepackage{xurl}}{} % add URL line breaks if available
\urlstyle{same}
\hypersetup{
  pdftitle={Research},
  colorlinks=true,
  linkcolor={blue},
  filecolor={Maroon},
  citecolor={Blue},
  urlcolor={Blue},
  pdfcreator={LaTeX via pandoc}}


\title{Research}
\author{}
\date{}
\begin{document}
\maketitle


Research

\textbf{Working Papers}

\href{Job_market_paper_Prajyna}{Estimating time-variation in matching
efficiency and match elasticity for the US labor market} (Job market
paper)

\ul{Conference presentations}: 2025 SEA (upcoming), 2025 SNDE, 2024 SEA,
NTxEC 2024, 2024 MEG, 2024 WEAI

The paper estimates time-varying matching efficiency and match
elasticity within a Cobb-Douglas matching function for the U.S. non-farm
sector, using a state-space model. It evaluates two specifications: (i)
no constant returns to scale (CRS) with time variation (my baseline
model) and (ii) CRS with time variation. I compare these two models with
the standard model without time variability, and the results show
significant time variation in both parameters, even under stochastic
volatility. Counterfactual vacancy estimates reveal that the non-CRS
baseline model fits the data best. This model suggests a gradual decline
in matching efficiency, indicating growing labor market inefficiencies.
Matching efficiency and match elasticity with respect to unemployment
are found to be procyclical. The procyclicality in efficiency can be
attributed to reduced sectoral reallocation, as seen in CPS data, driven
by structural economic shifts. Procyclical match elasticity points to
the nonlinear effect of labor market tightness on the job-finding rate.
The paper also finds a rise in the efficient unemployment rate after
COVID-19, as estimated by the baseline model.

\textbf{Estimated Output Gap in a Wage-Inflation Expectations Model}
(with Azharul Islam, Irina Panovska, Srikanth Ramamurthy)

We study how incorporating adaptive learning-based inflation
expectations for price and wage inflation can improve the performance of
Unobserved Components (UC) models both when it comes to inference about
the output gap and when it comes to forecasting performance. We use a
quasi-structural model where the statistical UC decomposition is
augmented with an inflation and a wage Phillips curve, and both the
output gap and price and wage inflation expectations are estimated
endogenously. To endogenize the expectations process, we incorporate
learning-based expectations and bivariate feedback in the learning
dynamics for wage and price inflation. Our model directly integrates the
expectations dynamics of the Hybrid New Keynesian Philips Curve while
also retaining the appealing statistical features of the UC framework,
allowing us to extract information about the output gap. Several
interesting sets of results stand out. First, we find conclusive
evidence of bivariate feedback in the dynamics of price and wage
inflation expectations. Second, while the perceived persistence of
inflation fell during the early stages of the COVID pandemic, it
increased sharply and substantially during the period 2021Q2-2022Q.
Third, the estimated output gap started decreasing in mid-2019 prior to
the pandemic, decreased sharply during the early stages of the pandemic,
and bounced back rapidly. When compared to models that do not include
information about wages and to models that do not endogenize inflation,
our approach tracks recession dates quite closely. Finally, including
information about the output gap and about the price and wage inflation
expectations process helps improve macroeconomic forecasts for GDP
growth, wage, and price inflation.

\textbf{Understanding unemployment dynamics}

\ul{Conference presentation:} NTxEC 2025 (upcoming)

In this paper, I revisit unemployment dynamics over the period
2003--2024, both in aggregate and across industries, using micro-level
CPS data from IPUMS. Transitions from non-participation to both
employment and unemployment play an important role. During the Great
Recession, the increase in unemployment reflected both stronger inflows
into unemployment (from employment and the labor force) and weaker
outflows into employment. In contrast, during COVID-19, the sharp
increase in unemployment was driven almost entirely by higher inflows
from employment into unemployment. The relative importance of different
flows for the cyclicality of unemployment also shifted over time.
Between 2006 and 2012, job separations and job-finding rates had effects
of comparable magnitude. However, in the subsequent period
countercyclical separations became the dominant driver of cyclical
fluctuations in unemployment, while the role of job finding diminished.
Finally, the variation in the transitions from nonparticipation to
unemployment accounted for 15--20\% of the cyclical variation in the
unemployment rate between 2013 and 2024. At an industry level, not all
industries are at steady state, and the paper looks at out-of-steady
state dynamics to decompose the variations in unemployment rate to the
labor market flows.

\hyperref[0]{Mexico's Labor Market through the Lens of the Beveridge
Curve} (with Luis Fernando Colunga-Ramos, Irina Panovska, Miroslava
Quiroga-Trevino, B. Elam Rodríguez-Alcaraz)

This paper presents the first Beveridge Curves for Mexico and its
regions, constructed from a newly compiled dataset on job vacancies.
Focusing on the post-COVID-19 period, the evidence reveals a sharp rise
in labor market slack at the onset of the pandemic, followed by a
gradual tightening from 2021 onward, with northern regions experiencing
a faster recovery. By 2023, vacancy rates had declined without a
corresponding increase in unemployment, suggesting a rebalancing of
labor demand across regions. Building on the vacancy--unemployment
relationship that defines the Beveridge Curve, the paper also develops
the first measure of labor market slack for Mexico: the vacancy to
unemployment (V/U) ratio, which is then used to analyze the relationship
between labor market conditions and inflation. In addition,
incorporating vacancy data into a Structural Bayesian VAR allows for the
identification of structural labor market shocks and the examination of
business cycle dynamics.The results indicate that Beveridge Curve
variables are primarily driven by labor market shocks: unemployment
dynamics are explained mainly by labor supply and matching efficiency
shocks, while wage bargaining shocks account for most of the variation
in vacancies at the national level, with notable heterogeneity across
regions.

\textbf{Understanding unemployment dynamics}

\ul{Conference presentation:} NTxEC 2025 (upcoming)

In this paper, I revisit unemployment dynamics over the period
2003--2024, both in aggregate and across industries, using micro-level
CPS data from IPUMS. Transitions from non-participation to both
employment and unemployment play an important role. During the Great
Recession, the increase in unemployment reflected both stronger inflows
into unemployment (from employment and the labor force) and weaker
outflows into employment. In contrast, during COVID-19, the sharp
increase in unemployment was driven almost entirely by higher inflows
from employment into unemployment. The relative importance of different
flows for the cyclicality of unemployment also shifted over time.
Between 2006 and 2012, job separations and job-finding rates had effects
of comparable magnitude. However, in the subsequent period
countercyclical separations became the dominant driver of cyclical
fluctuations in unemployment, while the role of job finding diminished.
Finally, the variation in the transitions from nonparticipation to
unemployment accounted for 15--20\% of the cyclical variation in the
unemployment rate between 2013 and 2024. At an industry level, not all
industries are at steady state, and the paper looks at out-of-steady
state dynamics to decompose the variations in unemployment rate to the
labor market flows.

I am a Ph.D.~candidate in Economics at the University of Texas at
Dallas. My research focuses on understanding the evolving trade-offs
between job vacancies and unemployment in the U.S. labor market. My
overall research focuses on on understanding the Beveridge curve
dynamics and therefore understanding the dynamics of unemployment. I
find that matching efficiency in the labor market has declined over time
since 2003. This had a negative effect on job finding probabilities and
thus has been one of the reason for increase in unemployment rate.
However, I also find that the effects of inflows on unemployment rate
have increased post Great Recession compared to the outflows out of
unemployment suggesting that the huge increase in inflows was the
primary reason for the shift in the Beveridge Curve during COVID and not
the decline in matching efficiency.

The link to my job market paper is
\href{Job_market_paper_Prajyna.pdf}{here}. My research focuses
unemployment dynamics, specifically how consumer beliefs about the
economy and inflation impact real economic activity, with a primary
emphasis on empirical macroeconomics and time series analysis, and a
secondary focus on econometrics.

Using data from the Current Population Survey (CPS), I examine the
sources of labor market inefficiencies and analyze the dynamics of
unemployment through the lens of labor market flows. In particular, I
study how the effects of these flows on unemployment have changed over
time.

In my job market paper link I contribute to the literature by estimating
time-varying matching efficiency within a state-space modeling framework
to explain shifts in the Beveridge curve. I find that matching
efficiency is pro-cyclical and that its decline during
downturns---evident through micro-level data---contributes to a rising
unemployment gap and a falling job-finding probability during
recessions.

You can find my \href{teachingPhil.qmd}{teaching philosophy} and
\href{researchPhil.qmd}{research statement} here.

\textbf{Contact Information}

If you'd like to learn more about my background, please feel free to
\href{Resume.qmd}{view my CV here}.




\end{document}
