% Options for packages loaded elsewhere
% Options for packages loaded elsewhere
\PassOptionsToPackage{unicode}{hyperref}
\PassOptionsToPackage{hyphens}{url}
\PassOptionsToPackage{dvipsnames,svgnames,x11names}{xcolor}
%
\documentclass[
  10pt,
]{article}
\usepackage{xcolor}
\usepackage[top=15mm,left=25mm,right=25mm,bottom=25mm,heightrounded]{geometry}
\usepackage{amsmath,amssymb}
\setcounter{secnumdepth}{-\maxdimen} % remove section numbering
\usepackage{iftex}
\ifPDFTeX
  \usepackage[T1]{fontenc}
  \usepackage[utf8]{inputenc}
  \usepackage{textcomp} % provide euro and other symbols
\else % if luatex or xetex
  \usepackage{unicode-math} % this also loads fontspec
  \defaultfontfeatures{Scale=MatchLowercase}
  \defaultfontfeatures[\rmfamily]{Ligatures=TeX,Scale=1}
\fi
\usepackage[]{libertinus}
\ifPDFTeX\else
  % xetex/luatex font selection
\fi
% Use upquote if available, for straight quotes in verbatim environments
\IfFileExists{upquote.sty}{\usepackage{upquote}}{}
\IfFileExists{microtype.sty}{% use microtype if available
  \usepackage[]{microtype}
  \UseMicrotypeSet[protrusion]{basicmath} % disable protrusion for tt fonts
}{}
\makeatletter
\@ifundefined{KOMAClassName}{% if non-KOMA class
  \IfFileExists{parskip.sty}{%
    \usepackage{parskip}
  }{% else
    \setlength{\parindent}{0pt}
    \setlength{\parskip}{6pt plus 2pt minus 1pt}}
}{% if KOMA class
  \KOMAoptions{parskip=half}}
\makeatother
% Make \paragraph and \subparagraph free-standing
\makeatletter
\ifx\paragraph\undefined\else
  \let\oldparagraph\paragraph
  \renewcommand{\paragraph}{
    \@ifstar
      \xxxParagraphStar
      \xxxParagraphNoStar
  }
  \newcommand{\xxxParagraphStar}[1]{\oldparagraph*{#1}\mbox{}}
  \newcommand{\xxxParagraphNoStar}[1]{\oldparagraph{#1}\mbox{}}
\fi
\ifx\subparagraph\undefined\else
  \let\oldsubparagraph\subparagraph
  \renewcommand{\subparagraph}{
    \@ifstar
      \xxxSubParagraphStar
      \xxxSubParagraphNoStar
  }
  \newcommand{\xxxSubParagraphStar}[1]{\oldsubparagraph*{#1}\mbox{}}
  \newcommand{\xxxSubParagraphNoStar}[1]{\oldsubparagraph{#1}\mbox{}}
\fi
\makeatother


\usepackage{longtable,booktabs,array}
\usepackage{calc} % for calculating minipage widths
% Correct order of tables after \paragraph or \subparagraph
\usepackage{etoolbox}
\makeatletter
\patchcmd\longtable{\par}{\if@noskipsec\mbox{}\fi\par}{}{}
\makeatother
% Allow footnotes in longtable head/foot
\IfFileExists{footnotehyper.sty}{\usepackage{footnotehyper}}{\usepackage{footnote}}
\makesavenoteenv{longtable}
\usepackage{graphicx}
\makeatletter
\newsavebox\pandoc@box
\newcommand*\pandocbounded[1]{% scales image to fit in text height/width
  \sbox\pandoc@box{#1}%
  \Gscale@div\@tempa{\textheight}{\dimexpr\ht\pandoc@box+\dp\pandoc@box\relax}%
  \Gscale@div\@tempb{\linewidth}{\wd\pandoc@box}%
  \ifdim\@tempb\p@<\@tempa\p@\let\@tempa\@tempb\fi% select the smaller of both
  \ifdim\@tempa\p@<\p@\scalebox{\@tempa}{\usebox\pandoc@box}%
  \else\usebox{\pandoc@box}%
  \fi%
}
% Set default figure placement to htbp
\def\fps@figure{htbp}
\makeatother





\setlength{\emergencystretch}{3em} % prevent overfull lines

\providecommand{\tightlist}{%
  \setlength{\itemsep}{0pt}\setlength{\parskip}{0pt}}



 


\usepackage{setspace}
\setstretch{1}
\usepackage{fancyhdr}
\usepackage{multicol}
\usepackage{lipsum}
\usepackage{hyperref}
\pagestyle{fancy}
\fancyhf{} % clear all header/footer fields
\rfoot{\thepage} % right footer with page number
\setlength{\multicolsep}{2pt plus 1.0pt minus 0.75pt}
\setlength{\columnsep}{3cm}
\usepackage{titlesec}
\titlespacing*{\section}{0pt}{0.5em}{0.3em}
\titlespacing*{\subsection}{0pt}{0.3em}{0.2em}
\makeatletter
\@ifpackageloaded{caption}{}{\usepackage{caption}}
\AtBeginDocument{%
\ifdefined\contentsname
  \renewcommand*\contentsname{Table of contents}
\else
  \newcommand\contentsname{Table of contents}
\fi
\ifdefined\listfigurename
  \renewcommand*\listfigurename{List of Figures}
\else
  \newcommand\listfigurename{List of Figures}
\fi
\ifdefined\listtablename
  \renewcommand*\listtablename{List of Tables}
\else
  \newcommand\listtablename{List of Tables}
\fi
\ifdefined\figurename
  \renewcommand*\figurename{Figure}
\else
  \newcommand\figurename{Figure}
\fi
\ifdefined\tablename
  \renewcommand*\tablename{Table}
\else
  \newcommand\tablename{Table}
\fi
}
\@ifpackageloaded{float}{}{\usepackage{float}}
\floatstyle{ruled}
\@ifundefined{c@chapter}{\newfloat{codelisting}{h}{lop}}{\newfloat{codelisting}{h}{lop}[chapter]}
\floatname{codelisting}{Listing}
\newcommand*\listoflistings{\listof{codelisting}{List of Listings}}
\makeatother
\makeatletter
\makeatother
\makeatletter
\@ifpackageloaded{caption}{}{\usepackage{caption}}
\@ifpackageloaded{subcaption}{}{\usepackage{subcaption}}
\makeatother
\usepackage{bookmark}
\IfFileExists{xurl.sty}{\usepackage{xurl}}{} % add URL line breaks if available
\urlstyle{same}
\hypersetup{
  colorlinks=true,
  linkcolor={blue},
  filecolor={Maroon},
  citecolor={Blue},
  urlcolor={Blue},
  pdfcreator={LaTeX via pandoc}}


\author{}
\date{}
\begin{document}


\begin{flushright}
{\large \textbf{Prajyna P. Barua}}\\[0.5em]
\href{mailto:barua.prajyna@gmail.com}{Email} \quad | \quad
\href{https://www.linkedin.com/in/prajynabarua}{LinkedIn}\\[0.5em]
+1 201-850-9213
\end{flushright}

My teaching philosophy is grounded in the belief that learning is not
simply the accumulation of knowledge, but the development of the ability
to apply, retain, and reproduce ideas in meaningful and lasting ways. I
view learning as an active, iterative process---one that requires
continuous self-evaluation, critical thinking, and the capacity to
connect abstract concepts to real-world issues.

I view teaching as a collaborative intellectual endeavor between
instructor and students. For students, the central challenge is to think
critically about what they learn and to grasp its broader significance.
For instructors, the task is to present complex material in a way that
is both clear and intellectually rigorous, while fostering a classroom
environment that supports curiosity, confidence, and ongoing engagement.
Incorporating student feedback is essential, making the teaching process
dynamic and often reflective, requiring continual adaptation and
introspection. This understanding came because of getting a range of
experience form teaching undergraduate classes (both small and large
class size), the graduate level bootcamp for incoming masters and
Ph.D.~cohort and being a teaching assistant for a range of classes
including forecasting and quantitative focused classes.

In my classroom, I aim to simplify technical content through real-world
examples, discussion-based learning, and intuitive explanations. I
intentionally maintain a medium pace in lectures, recognizing that
students learn at different speeds. I frequently use the whiteboard
alongside slides---not only to slow the lecture tempo, but also to keep
students focused and engaged. This approach allows me to observe
students' reactions, and when I sense a dip in attention, I introduce an
active learning question or real-world example to bring them back into
the discussion.

I begin each class by revisiting key points from the previous
session---particularly when the material is cumulative. This helps
students orient themselves, even if they haven't reviewed the last
lecture, and ensures a smoother transition into new content. Before
introducing a topic, I also contextualize its importance. For instance,
when teaching GDP and national accounting, I might open with the
question: ``What does it mean for a country to be economically
well-off?'' This framing helps students understand the relevance of what
they are about to learn.

I stay current with economic developments through research, policy
discussions, and academic conferences. Bringing this knowledge into the
classroom allows me to highlight real-time applications of economic
theory. I've found that students often respond enthusiastically---asking
questions about methodology, data sources, and research careers. One
student approached me to ask how to pursue a career in research and what
skills were needed, which I found particularly rewarding.

To provide consistent feedback and keep students engaged, I introduced
short in-class quizzes at the end of each chapter. To encourage
discussion among students, I usually ask them to discuss with their
peers and then mark their answers so that they do not attempt these
quizzes individually and are able to get others' opinion before
selecting their answers. These quizzes are followed by open discussion,
allowing students to reflect on their responses and learn from their
mistakes. Although they are self-graded to reduce pressure, I ask how
many students answered at least three out of five questions correctly.
This informal poll encourages self-assessment and helps me gauge overall
comprehension, allowing me to adjust the pace and revisit key concepts
when needed. This approach has helped increase both engagement and
attendance.

I also emphasize practical skills by incorporating hands-on data
exercises. When teaching macroeconomic indicators such as GDP,
inflation, unemployment, and exchange rates, I guide students through
how these statistics are constructed and how to access them using
real-world data. This reinforces the connection between theory and
policy and empowers students to pursue independent analysis.

Some of the most rewarding teaching moments occur when students begin to
think independently, ask insightful questions, or relate concepts to
broader economic issues. These instances often spark peer-to-peer
dialogue and lead to stronger retention of material. Ultimately, I
believe that effective teaching is not simply about transferring
knowledge---it is about instilling intellectual habits that will serve
students long after they leave the classroom.

To support exam preparation, I provide students with practice questions
approximately ten days in advance. This strategy encourages early
engagement with the material and offers bonus credit for those who
complete the questions prior to the review session. During the review,
we discuss the solutions in depth, focusing on common pitfalls and
reinforcing key concepts. I deliberately choose not to post the answers
online, which increases attendance and ensures students come prepared to
participate. These sessions are most effective when students have made
an effort to engage beforehand, transforming the review into an
interactive learning experience rather than a passive lecture.
Additionally, this provides a valuable opportunity to support students
who may be struggling by offering them a chance ask questions and
improve their performance before the exam.

In addition to my undergraduate teaching, I served as an instructor for
the School of Economic, Political and Policy Sciences Math Boot Camp,
which I co-designed and co-taught with a colleague. I was fortunate
enough to be selected among other students for teaching in the EPPS Boot
Camp. The program was developed for incoming Ph.D.~and Master's students
in EPPS---many of whom had limited exposure to advanced mathematics. Our
goal was to reinforce essential mathematical tools while bridging the
gap between theory and empirical application.

We focused on helping students understand mathematical relationships in
the context of data and modeling assumptions. For instance, we
demonstrated how matrix algebra forms the foundation of vector
autoregressive (VAR) models, and discussed how different probability
distributions are selected based on data properties. To deepen their
engagement, we assigned research papers in advance and used them as the
basis for class discussions---encouraging students to evaluate the
assumptions and practical relevance of each method.

I was also part of the virtual exchange program for Contemporary
Macroeconomic Policy as teaching assistant. The virtual collaboration
was between students from UTD and University of Marburg, Germany. We
created a comprehensive agenda outlining key dates, expectations,
grading criteria, and resource links to support students throughout the
collaboration. Teams were thoughtfully formed to foster cohesion, and a
WhatsApp group---established with consent---enabled informal
communication without instructor interference. We encouraged
participation through suggested ice-breakers and used Padlet to
facilitate easy video sharing. Ongoing coordination between faculty and
teaching assistants on both sides ensured the collaboration ran
smoothly. This experience underscored how intentional design and
responsiveness can create an inclusive, dynamic learning environment
incorporating different geographies. In the business and economic
forecasting class, where I served as a teaching assistant, students
developed research projects that applied various quantitative and
forecasting techniques and evaluated the performance of their models.

In all of my teaching, I aim to inspire curiosity, build confidence, and
equip students with both the theoretical grounding and practical tools
they need to succeed. I am happy to teach any courses as needed,
however, based on my research and teaching experience, I am particularly
well equipped to teach principles of microeconomics and macroeconomics,
intermediate microeconomics and macroeconomics, international trade,
business statistics, econometrics and mathematics.

If given the opportunity to work as an Assistant Professor of Economics
at {[}\emph{Institution}{]}, I look forward to contributing to a
collaborative learning environment with both faculty and students. I aim
to integrate research and data analytics into my courses, and mentor
students by engaging with them in my research projects, and embrace
opportunities to learn alongside them. {[}\emph{Last sentence is
specific to the program/institution and addresses how I will contribute
to the growth and development of the teaching population.}{]}




\end{document}
